% !TEX program = xelatex
\documentclass[a4paper]{article}
\usepackage{amsmath}
\usepackage{amsthm}
\usepackage[left=1.8cm,right=1.8cm,top=2.2cm,bottom=2.0cm]{geometry}
\usepackage{ctex}
\usepackage{enumerate}
\usepackage{fancyhdr}
\usepackage{xpatch}
\usepackage{graphicx} 
\usepackage{float} 
\usepackage{subfigure} 
\usepackage{amsfonts}
\usepackage{mathtools}
\usepackage{framed}
\usepackage{multicol}
\usepackage{listings}
\usepackage{hyperref}
\usepackage{tikz}
\usetikzlibrary{automata,positioning}
\theoremstyle{definition}
\newtheorem*{solution*}{\textbf{Solution:}}
\newtheorem*{proof*}{\textbf{Proof:}}
\newtheorem{theorem}{Theorem}[subsection]
\newtheorem{definition}{Definition}[subsection]
\newtheorem{lemma}{Lemma}[subsection]
\makeatletter
\usepackage{listings}% http://ctan.org/pkg/listings
\lstset{
  basicstyle=\ttfamily\tiny,
  mathescape,
}
\AtBeginDocument{\xpatchcmd{\@thm}{\thm@headpunct{.}}{\thm@headpunct{}}{}{}}
\makeatother

\pagestyle{fancy}
\renewcommand{\baselinestretch}{1.15}

\usepackage{paralist}
\let\itemize\compactitem
\let\enditemize\endcompactitem
\let\enumerate\compactenum
\let\endenum
\let\description\compactdesc
\let\enddescription\endcompactdesc

% shorten footnote rule
\xpatchcmd\footnoterule
  {.4\columnwidth}
  {1in}
  {}{\fail}

\title{CS 131 Compilers: Discussion 13: Garbage Collection}
\author{\textbf{杨易为}~~\textbf{季杨彪}~~\textbf{尤存翰} \\ \texttt{ \{yangyw,jiyb,youch\}@shanghaitech.edu.cn}}


\begin{document}
\maketitle
\section{Resource Acquisition Is Initialization}
RAII, which is the compiled time garbage collector first introduced in this \href{thread}{http://www.open-std.org/jtc1/sc22/wg21/docs/papers/2008/n2670.htm}, guarantees that the resource is available to any function that may access the object (resource availability is a class invariant, eliminating redundant runtime tests). It also guarantees that all resources are released when the lifetime of their controlling object ends, in reverse order of acquisition. Likewise, if resource acquisition fails (the constructor exits with an exception), all resources acquired by every fully-constructed member and base subobject are released in reverse order of initialization. This leverages the core language features (object lifetime, scope exit, order of initialization and stack unwinding) to eliminate resource leaks and guarantee exception safety. Another name for this technique is Scope-Bound Resource Management (SBRM), after the basic use case where the lifetime of an RAII object ends due to scope exit.

RAII can be summarized as follows:
\begin{enumerate}
  \item encapsulate each resource into a class, where
  \begin{enumerate}
    \item the constructor acquires the resource and establishes all class invariants or throws an exception if that cannot be done,
    \item the destructor releases the resource and never throws exceptions;
    
  \end{enumerate}
  \item always use the resource via an instance of a RAII-class that either
  \begin{enumerate}
    \item has automatic storage duration or temporary lifetime itself, or
\item    has lifetime that is bounded by the lifetime of an automatic or temporary object
  \end{enumerate}
  Move semantics make it possible to safely transfer resource ownership between objects, across scopes, and in and out of threads, while maintaining resource safety.
\\
  Classes with open()/close(), lock()/unlock(), or init()/copyFrom()/destroy() member functions are typical examples of non-RAII classes:
\end{enumerate}
\end{document}
