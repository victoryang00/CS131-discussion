 % !TEX program = xelatex
\documentclass[a4paper]{article}
\usepackage{amsmath}
\usepackage{amsthm}
\usepackage[left=1.8cm,right=1.8cm,top=2.2cm,bottom=2.0cm]{geometry}
\usepackage{ctex}
\usepackage{enumerate}
\usepackage{fancyhdr}
\usepackage{xpatch}
\usepackage{graphicx} 
\usepackage{float} 
\usepackage{subfigure} 
\usepackage{amsfonts}
\usepackage{mathtools}
\usepackage{framed}
\usepackage{multicol}
\usepackage{listings}
\usepackage{hyperref}
\usepackage{tikz}
\usetikzlibrary{automata,positioning}
\theoremstyle{definition}
\newtheorem*{solution*}{\textbf{Solution:}}
\newtheorem*{proof*}{\textbf{Proof:}}
\newtheorem{theorem}{Theorem}[subsection]
\newtheorem{definition}{Definition}[subsection]
\newtheorem{lemma}{Lemma}[subsection]
\makeatletter

\AtBeginDocument{\xpatchcmd{\@thm}{\thm@headpunct{.}}{\thm@headpunct{}}{}{}}
\makeatother

\pagestyle{fancy}
\renewcommand{\baselinestretch}{1.15}

\usepackage{paralist}
\let\itemize\compactitem
\let\enditemize\endcompactitem
\let\enumerate\compactenum
\let\endenumerate\endcompactenum
\let\description\compactdesc
\let\enddescription\endcompactdesc

% shorten footnote rule
\xpatchcmd\footnoterule
  {.4\columnwidth}
  {1in}
  {}{\fail}

\title{CS 131 Compilers: Discussion 5: Shift-Reduce Parsing}
\author{\textbf{杨易为}~~\textbf{季杨彪}~~\textbf{尤存翰} \\ \texttt{ \{yangyw,jiyb,youch\}@shanghaitech.edu.cn}}


\begin{document}
\maketitle
% \section{DFA and NFA}
% \subsection{Introduction}
\section{Shift-Reduce Parsing the Lambda Calculus.}
 We'll look again at the lambda calculus grammar:

 \begin{verbatim}
  var  : ID ;
  expr : var
       . ‘(’ ‘l’ var ‘.’ expr ‘)’
       | ‘(’ expr expr ‘)’ ;
  \end{verbatim}
\begin{enumerate}
    \item Is this grammar $\mathrm{LL}(1) ?$
    \item We'll now use the following LR(1) parsing table to parse some strings with this
    grammar.
    \item Is this grammar LR(0)?
\end{enumerate}
\textbf{Answer:}


\section{Altering the Lambda Calculus.}
Suppose we want to add an optional extension that allows raising avarto a power.We define the grammar as
\begin{verbatim}
  expr : var 
       | var '-' NUM
       | ‘(’ ‘lambda’ var ‘.’ expr ‘)’
       | ‘(’ expr expr ‘)’ ;
  var  : ID ;
  \end{verbatim}
  \begin{enumerate}
    \item Is this grammar LR(0)?
    \item Which state in the parsing table would we need to modify to parse this grammar?
  \end{enumerate}

\textbf{Answer:}

\section{Stack in Shift-Reduce Parsing.}
Suppose  it  is  given  that  shift-reduce  parsing  is  equivalent  to  finding  the  rightmostderivation in reverse.  Prove that during shift-reduce parsing, we can only reduce thetopmost items in the stack (i.e.  we don’t need to worry about reducing something inthe middle; hence the usage of a stack is justified)

\section{Exercises}
Consider the following CFG, which has the set of terminals $T=\{\mathbf{a}, \mathbf{b}\}$

$$\begin{aligned} S & \rightarrow X \mathbf{a} \\ X & \rightarrow \mathbf{a} \mid \mathbf{a} X \mathbf{b} \end{aligned}$$
\begin{enumerate}
\item Construct a DFA for viable prefiexes of this grammar using LR(0) items.
\item Identify a shift-reduce conflict in this grammar under the SLR(1) rules.
\item Assuming that an SLR(1) parser resolves shif-reduce confilcts by choosing to shift, show the operation of such a parser on the input string \textbf{aaba}.
\item Suppose that the production $X\leftarrow \epsilon$ is added to this grammar. Identify a reduce-reduce conflict in the resulting grammar under the SLR(1) rules.
\end{enumerate}
\textbf{Answer:}



\end{document}
